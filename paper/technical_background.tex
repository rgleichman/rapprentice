\emph{Trajectory transfer} is an approach to learning from expert demonstrations~\cite{Schulmanetal_ISRR2013}. The trajectory transfer
algorithm is given a current scene, $s_{test}$, demonstration scene, $s_{demo}$ and a demonstration trajectory,  
$t_{demo}$, as input. We assume that the scenes are made up of matched points in $\mathbb{R}^3$. The first step is to 
find a function, $f^*:\mathbb{R}^3 \rightarrow \mathbb{R}^3$, as the solution to the following optimization problem:
\begin{equation}\min_f \sum_i ||s_{test}^{(i)} - f(s_{demo}^{(i)})||^2 + C\int dx ||D^2(f)||^2_{Frob}.\label{eq:tps}\end{equation}
The minimizing $f$ will be a Thin Plate Spline, and can be expressed as a linear combination of basis
functions about the correspondence points~\cite{Wahba_TPS1990}. $C$ is a hyper-parameter that trades off
between goodness of fit and the curvature of the function. The solution to this optimization can be computed
as the solution to a linear system of equations.

Given a warping, $f^*$, between the demo and test scenes, we take each pose from the demo trajectory and pase it through
$f^*$. Poses are transferred by mapping coordinate frames through the jacobian of $f^*$. The trajectory that results from this
is used to guide a motion planner that finds a similar feasible trajectory. This trajectory is executed in the test scene.
In the case where correspondences are not known initially, one can use TPS-RPM, an approach that jointly finds
correspondences and a mapping between them by alternating between estimating correspondences and solving for 
a thin plate spline\cite{Chui_CVIU2003}.

Schulman et al.~\cite{Schulmanetal_ISRR2013} provide some intuition for scenarios where this approach is likely
to succeed. They assume a cost function, $L$, on states and trajectories, a reasonable option might be 0-1 loss, depending
on whether the trajectory successfully executes a desired manipulation in a given state.  Then we can justify warping the
state $s$ and the trajectory $t$ 
in the case where $L(s, t)~=~L(f(s), f(t))$. Essentially, the of a manipulation is preserved under a class of transformations,
thus, we can successfully transform a state and trajectory and maintain the relation that the manipulation succeeds. The set of functions
that have this property define a set of states that a particular demonstration trajectory can transfer to.

A final aspect of this approach is the incorporation of multiple trajectories. Given a library of trajectories,
one can increase the number of states that can be generalized to. This allows an expert to demonstrate steps of a complex task
which can be sequenced at test time. This can make trajectory transfer more robust and reliable, as an expert can also
include demonstrations to recover from common failures. Current approaches use the nearest-neighbor with respect to registration
cost (the value of the optimization problem defined in (\ref{eq:tps})). This corresponds to modeling the set of states a demonstration
can generalize to---that is states for which $L$ is invariant to the TPS warping found by TPS-RPM---as a hyper-sphere 
in a high-dimensional space where this registration cost is a distance function.


