A large challenge in applying standard manipulation and planning techniques to
deformable object manipulation is that of tractable modelling. Deformable object 
are often characterized by high-dimensional, continuous state-action spaces. Model-based 
planning has yet to scale up to the task of efficient planning in this
setting.

Recent work have gained traction on this problem through the technique of 
learning from demonstrations~\cite{Schulmanetal_ISRR2013,Schulmanetal_IROS2013}.
These results are achieved through \emph{trajectory transfer}, where a demonstration
trajectory is generalized to fit a new scenario. Trajectory transfer finds a non-rigid
registration is found an example scene and the current scene and that is
used to modify the demonstration trajectory so that it better fits the scene at hand.
This method of transfer is model-free and obviates the need to plan in complicated
and intractable models of deformable objects. Trajectory transfer has demonstrated 
state-of-the-art performance for knot-tying and suturing.

An important aspect of these strategies is the incorporation of multiple demonstrations
into the process. By increasing the amount of instruction, it becomes possible to do more
tasks. Additionally, demonstrations can take the form of steps in a task and can be
order or recombined to further increase the set of possible successful manipulations.

However, an important problem remains: how should we pick which trajectory to transfer
from a library of options given an input scene? Incorrect selection may lead us
to fail at a task which would otherwise be possible for the correct selection of 
trajectories.

Schulman et al. propose to use nearest-neighbors with registration cost (a measure of the goodness of 
fit for the registration) as a similarity measure to solve this problem. In this paper,
we consider the problem of effectively learning to pick the trajectory to generalize
from experience.

We frame this problem in terms of manifold learning with respect to the state space of our object.
Given a manipulation task, $m$, and demonstration state-trajectory pair, $d, t$, there is region of state space, $S$, 
such that $t$ will perform $m$ successfully when transferred to any $s \in S$. 
In this framework, the nearest-neighbor selection rule represents $S$ with the singlton $d$ and chooses the 
trajectory which minimizes distance to an known example from $S$. \dhm{This is pretty sloppy, and probably too
detailed for the intro but I'm including it so we have it down on paper.} A natural way to extend this approach uses successful traces of trajectory transfer
to draw new states from $S$.

The contributions of this paper are as follows: {\bf(i)} we frame the problem of selecting a trajectory to transfer
from a library as one of estimating distance to a manifold; {\bf(ii)} we propose a method for building a model-free
representation of the manifold associated with a demonstration from successful traces of execution; {\bf(iii)} we show
how this approach can be leveraged to improve finding correspondences with new scenes; {\bf(iv)} we demonstrate
the effectiveness of this approach by showing an improvement of \dhm{Number here} over the nearest-neighbor selection
strategy in a simulated knot-tying task.

\dhm{We need a name}

\dhm{Figures: illustration of segment tree; Graphs/Table showing performance of NN/No Correspondence/Best C-Forest; 
     Graph showing performance of C-Forest as fn of number of training examples; Illustration of manifold idea}




