A large challenge in applying standard manipulation and planning techniques to
deformable object manipulation is that of tractable modelling. Deformable object 
are often characterized by high-dimensional, continuous state-action spaces. Model-based 
planning has yet to scale up to the task of efficient planning in this
setting.

Recent work have gained traction on this problem through the technique of 
learning from demonstrations~\cite{Schulmanetal_ISRR2013,Schulmanetal_IROS2013}.
These results are achieved through \emph{trajectory transfer}, where a demonstration
trajectory is generalized to fit a new scenario. Trajectory transfer finds a non-rigid
registration between an example scene and the current scene that trades off between
goodenss-of-fit and the curvature of the registration. 
This method of transfer is model-free and obviates the need to plan in complicated
and intractable models of deformable objects. Trajectory transfer has demonstrated 
state-of-the-art performance for knot-tying and suturing.

An important aspect of these strategies is the incorporation of multiple demonstrations
into the process. By increasing the amount of instruction, it becomes possible to do more
tasks. Additionally, demonstrations can take the form of steps in a task and can be
order or recombined to further increase the set of possible successful manipulations.

However, an important problem remains: how should we pick which trajectory to transfer
from a library of options given an input scene? Incorrect selection may lead us
to fail at a task which would otherwise be possible for the correct selection of 
trajectories. Furthermore, how can we use experience to improve trajectory transfer 
both at the level of selecting a trajectory to transfer and at the level of transferring
a single trajectory.

The contributions of this paper are as follows: {\bf(i)} we frame the problem of selecting 
a trajectory to transfer from a library as one of modelling the \emph{transfer set} for a 
demonstration: the set of states under which a trajectory transfer method respects 
condition for successful completion of a manipulation task; {\bf(ii)} 
we propose a general method for learning these sets from successful traces of 
execution; {\bf(iii)} we show that this method be used to bootstrap new demonstrations
increase the size of the transfer set associated with a trajectory; {\bf(iv)} we demonstrate
the effectiveness of these improvements in knot-tying task and achieve an improvement 
of \dhm{Number here} over the transfer method described in Schulman et al.~\cite{Schulmanetal_ISRR2013}.



