In this section,
%\subsection{Transfer Manifolds}
%\subsection{Trajectory Selection and a Nearest-Manifold Query}
%\subsection{Learning Segment Trees}

As a motivating example, consider the problem of tying knots in ropes. The process of tying a knot can be broken down into several steps where each step consists of a simple action of grabbing the rope in one or two places, moving the grippers, and then releasing the rope. We call each step a segment. See \ref{knots.fig} for some example knots and tying segments. To help the robot tie the knot a human records several expert demonstrations by manually guiding the trajectory of the grippers and commanding the robot when to open and close the grippers. For each step or segment during the task demonstration of tying the knot, the robot records the initial state of the rope in the form of a pointcloud, the trajectory of the grippers, and the gripper closing and opening events. Look at \ref{trajectories.fig} to see some of the paths of the gripper trajectories.

After the demonsrations are recorded, the robot can now test the demonstations on new initial rope states. First, the robot chooses a demonstration to use. This could be done either willy-nilly or via a sophisticated algorithm. The robot now faces the problem of applying the gripper trajectory from the demonstration rope to the test rope. Obviously, simply executing the demonstration trajectory will not work since the test rope is in a different position in the workspace. To solve this problem, the robot finds a non-rigid transformation between the chosen demonstration segment and the test rope. The transformation is found using the TPS-RPM algorithm as described in the \ref{Technical Background} section. The figure \ref{transform.fig} shows an example demonsration segment, a test rope, and the non-rigid transformation from the demonstration rope to the test rope. When this transformation is applied to the demonsration rope, the transformed rope should be very close to the test rope. The robot now knows where to grip, and it applies this same transformation to the demonstration gripper trajectory. The robot then executes the the transformed gripper trajectory. This process of choosing a demonstration, finding and applying a transformation, and executing the transformed trajectory for every step in the knot tying process. Eventually, the robot will either tie a knot and celebrate the skill of its human demonstrators and programmers, fail in executing a segment (eg. it failed to grip the rope), or it gets frusturated and gives up after executing several segments.
