In this section,
%\subsection{Transfer Manifolds}
%\subsection{Trajectory Selection and a Nearest-Manifold Query}
%\subsection{Learning Segment Trees}

As a motivating example, consider the problem of tying knots in ropes. The process of tying a knot can be broken down into several steps where each step consists of a simple action of grabbing the rope in one or two places, moving the grippers, and then releasing the rope. We call each step a segment. See \fig{knots} for some example knots and tying segments. To help the robot tie the knot a human records several expert demonstrations by manually guiding the trajectory of the grippers and commanding the robot when to open and close the grippers. For each step or segment during the task demonstration of tying the knot, the robot records the initial state of the rope in the form of a pointcloud, the trajectory of the grippers, and the gripper closing and opening events.
